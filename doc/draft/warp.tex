\section{Warping}
\label{sect:warping}

The key idea of this paper is that when transactions run
awry into an inconsistent state, rather than aborting them and
starting again from the beginning, it may be possible to
\emph{correct} the shared/local state directly by directly
modifying the state. Thus, the program continues as if it had not
gone down the bad path to the inconsistent state.

We now formalize a simple version where, from an inconsistent state,
the system warps to a state that was reachable in a serial
interleaving. We later show this can be easily generalized to allow
warping to more subtle states (such as those described by the
Push/Pull model~\cite{}). These \emph{super}-serial (yet still
serializable) allow for more possible destinations, making it more
likely that a destination will be available when an inconsistent state
is reached.


\paragraph{Unconstrained transition system.}

We begin with a generic transition system $\xrightarrow{u}$ in which threads may
interleave their effects however they please:
%{\bf Unconstrained Machine Rules} $\xrightarrow{u}$
$$
\infer[\text{\sc UFin}]{ 
  \As_1 \cdot (c,\sigma) \cdot \As_2, G  \xrightarrow{u}
  \As_1 \cdot \As_2, G 
}{
  \nothing{c}
}
$$
$$
\infer[\text{\sc UFork}]{ 
  \As_1 \cdot (c,\sigma) \cdot \As_2, G  \xrightarrow{u}
  \As_1 \cdot (c_2,\sigma) \cdot (c',\sigma) \cdot \As_2, G 
}{
  \tstep{c_1}{\fork{c}}{c_2}
}
$$
$$
\infer[\text{\sc ULocal}]{ 
  \As_1 \cdot (c,\sigma) \cdot \As_2, G  \xrightarrow{u}
  \As_1 \cdot (c',\sigma') \cdot \As_2, G 
}{
  \tstep{c}{\local{R}}{c'} & R\ \sigma\ \sigma'
}
$$
$$
\infer[\text{\sc UTxn}]{ 
  \As_1 \cdot (\tx{c_1},\sigma) \cdot \As_2, G  \xrightarrow{u}
  \As_1 \cdot (\tx{c_2},\sigma') \cdot \As_2, G\cdot \langle m,\sigma,\sigma' \rangle
}{
  \step{c_1}{m}{c_2} &
  G \allows \langle m,\sigma,\sigma' \rangle
}
$$
The semantics is a relation
$\xrightarrow{u}$ over pairs consisting of a list of concurrent
threads $\As$ and a shared state $\OPL$. 
A single thread $(c,\sigma)\in\As$ is a code $c$ and local state $\sigma$. 

The unconstrained machine can take a {\sc UFin} step when there is a thread
$(c,\sigma)$ that can complete, \ie~$\nothing{c}$.
%
The {\sc UFork} rule allows a new thread
$(c',\sigma)$ to be forked from thread $(c,\sigma)$.
%
The {\sc ULocal} rule involves manipulating the thread-local state
$\sigma$ to $\sigma'$.
%
Finally, the {\sc UTxn} rule allows a thread executing transaction
code $\tx{c_1}$ to take a single step to $c_2$, applying the effects
of $m$ directly to the shared log $G$.

%\subsection{Serializable Transition Systems}

\paragraph{Corrective Warping and Committing.}
A simple version of warping (which we have implemented in
Section~\ref{sec:impl}) builds on an unconstrained transition system
$\xrightarrow{u}$ by adding a special {\sc WarpCmt} rule. This rule
attempts to perform a warp---
replacing the current state with state that would have been reached in
an atomic interleaving and then committing---or else, aborts. This is
already more expressive than all existing notions of transactions
which, in the face of an inconsistent state perform a (potentially
partial) abort.

\red{add the fact that each thread remembers the $\Ts,G$ where they began}

$$
\infer[\text{\sc WarpCmt}]{ 
  \As_1 \cdot ((\tx{c_1},\red{c_2}),\sigma) \cdot \As_2, G  \xrightarrow{u}
  \As'_1 \cdot (c'_2,\sigma') \cdot \As'_2, G'
}{
  \nothing{c_1} &
   \text{\red{Warp}}(  \As_1 \cdot (c_2,\sigma) \cdot \As_2, G,
  \As'_1 \cdot (c'_2,\sigma') \cdot \As'_2, G')
}
$$


\red{how to jump to an atomic state? you haven't committed yet.}

\red{reference state issue.}

\red{conditions on warping - consistency.}




% \red{\bf ALL THIS MAY BE DEAD}

% \begin{definition}[Serializable transition system]
% For all $\Ts,\OPL$, we say that a fragment 
% $\xrightarrow{ser}$ of transition system $\xrightarrow{}$ is
% serializable provided that 
% $\xrightarrow{ser}$ simulates $\xrightarrow{a}$.
% \end{definition}

% \red{explain what simulates means}

% As previously shown, the Push/Pull transition system is such a
% serializable transition system~\cite{PMPY}.

% \begin{definition}[Good configuration]
% A configuration $\Ts,\OPL$ is \emph{good} provided that it is reachable in
% some serializable transition system $\xrightarrow{ser}$
% \end{definition}


% \newcommand\rc[2]{\textsf{rc}(#1,#2)}
% \begin{definition}[Reference configuration]
% For a given transition system with initial configuration $\Ts_0,G_0$,
% we define a \emph{reference configuration}, denoted
% $\rc{\Ts,G}$ to be given by the fixpoint
% $$
% \infer{\rc{\Ts_0}{G}}{}
% \qquad
% \infer{\rc{\Ts'}{G''}}{
%    \rc{\Ts}{G}
% &  \Ts,G \xrightarrow{ser}^{*} \Ts',G'
% &  G'' \opeq G'
% }
% $$
% \end{definition}

% \paragraph{Warping correctness.} The goal of warping is simple. For a reference
% configuration $\Ts_0,G_0$, when we have 
% $\Ts_0,G_0 \xrightarrow{u}^{*} \Ts,G$, we wish to
% \[ \begin{array}{rl}
%    \text{Find } \Ts',G'' \text{ such that}&
%       \Ts_0,G_0 \xrightarrow{s} \Ts',G'\\
%   \text{and}&G' \opeq G''
% \end{array} \]

% \paragraph{Warping utility.}
% In particular, we would like to find a $\Ts',G''$ that can be 
% \emph{easily computed} from $\Ts,G$.




%%% Local Variables: 
%%% mode: latex
%%% TeX-master: "paper"
%%% End: 
