\section{OLD Preliminaries}

In this section we lay out some initial definitions. We will work with
the Push/Pull model of transactions, which is a fairly expressive model.
As we will see, it provides a nature model for laying
out a general notion of trace warping.

\paragraph{The Push/Pull Model.} We now summarize the Push/Pull
model. For a more full description, please see~\cite{pmpy}.
%
We work with shared object \emph{operations}, denoted $\theOP$,
consisting of a method name $m$, thread-local pre/post state $\sigma,
\sigma'$ (for arguments and return values)
%, a vector of arguments $\vec{a}$, a vector of return values
%$\vec{r}$,
and the operation's unique ID $id$.

The Push/Pull model is based purely on operation \emph{logs}, which
are lists of operations:
\[
  \OPL = [ \numOP{1},  ..., \numOP{n}]
\]
The \emph{sequential specification}  is given as a predicate on logs:
$\allowed{\OPL\cdot \theOP}$ or, alternatively, we may write 
$\OPL \allows \theOP$.

Threads keep a local log (or ``view'') and there is a single shared
log. The shared log keeps a flag next two each operation, indicating
whether the operation belongs to a committed transaction:
%
\[\begin{array}{rl}
  G = [ &(\numOP{1},\gCommitted),  \\
   &...\\
                &(\numOP{n},\gUncommitted)]
\end{array} \]

Each thread is a triple $\CSL$, where $c$ is the code to be executed,
$\sigma$ is the local state, and $L$ is the local log. The local log has a
flag next to each entry, indicating whether the operation is local
($\lUnpushed$), has been {\sc push}ed to the shared view ($\lPushed$) or
is an operation {\sc pull}ed from the shared log ($\lPulled$) and
belonging to another thread:
\[\begin{array}{rl}
  L = [ &(\numOP{1},\lPulled),  \\
   &...\\
                &(\numOP{n},\lUnpushed)]
\end{array} \]
%
The Push/Pull model is parametric over the reduction semantics of the
code $c$. To accomplish this, we rely on a fixpoint called $\textsf{step}$
and say that 
 $(m,\sigma',c') \in \step{\sigma,c}$
if code $c$ can be reduced from state $\sigma$ 
\emph{without executing any operation methods} to a next reachable
method $m$ and the remaining code/state $c'$ and $\sigma'$.

Finally, a configuration in the Push/Pull model is $\Ts,G$ where $G$
is the shared log and $\Ts = \numCSL{1}\cdot \cdots \cdot \numCSL{n}$
are the transactions

% \item
% precongruence relation between logs: $\OPL_1 \opeq \OPL_2 $.\\

\begin{definition}[Precongruence $\opeq$] For all $\OPL_1, \OPL_2$,
$$
\infer={\OPL_1 \opeq \OPL_2} 
   {  \allowed{\OPL_1} \Rightarrow \allowed{\OPL_2}
     & \forall \op.\   (\OPL_1 \cdot \op) \opeq (\OPL_2 \cdot \op)}
$$
Note that we use a double-line here to indicate greatest fixpoint.
\end{definition}

Intuitively, the above 
 coinductive fixpoint says that for all seq of ops applied after $\OPL_1$, the results match what would be observed (or is allowed) by $\OPL_2$

% \item
% obs. equiv
% $\OPL_1 \obseq \OPL_2 \;\;\equiv\;\; \OPL_1 \opeq \OPL_2 \wedge \OPL_2 \opeq \OPL_2$

% \item steps:
% $$
% \infer[\text{\sc fin}]{ \Ts_1 \cdot \CSL\cdot \Ts_2, \OPL \pmpyreduce  \Ts_1 \cdot\Ts_2, \OPL}{\nothing{c}}
% \hspace{0.1in}
% \infer[\text{\sc step}]{ \Ts_1 \cdot \CSL\cdot \Ts_2, \OPL \pmpyreduce  \Ts_1 \cdot \CpSpLp\cdot \Ts_2, \OPL'\cdot\theOP}
% {(c',m) \in \step{c} & \allowed{\OPL\cdot\theOP}}
% $$


%\begin{theorem}[Serializability of Push/Pull~\cite{pmpy}]
%The Push/Pull model is serializable.
%\end{theorem}

